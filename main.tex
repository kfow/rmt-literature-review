%% ================================================================================
%% This LaTeX file was created by AbiWord.                                         
%% AbiWord is a free, Open Source word processor.                                  
%% More information about AbiWord is available at http://www.abisource.com/        
%% ================================================================================

\documentclass[a4paper,portrait,12pt]{article}
\usepackage[latin1]{inputenc}
\usepackage{calc}
\usepackage{setspace}
\usepackage{fixltx2e}
\usepackage{graphicx}
\usepackage{multicol}

\usepackage{natbib}
%\usepackage[normalem]{ulem}
%% Please revise the following command, if your babel
%% package does not support en-US
%\usepackage{babel}
\usepackage{color}
\usepackage{hyperref}
\usepackage{ifthen}
\let\oldcite=\cite
\renewcommand\cite[1]{\ifthenelse{\equal{#1}{NEEDED}}{\ensuremath{^\texttt{[citation~needed]}}}{\oldcite{#1}}}
\author{Kelvin Fowler, 2083905f}
 
\title{The Application of Word Embeddings in Information Retrieval}

\begin{document}

\setlength{\oddsidemargin}{0.9847in-1in}
\setlength{\textwidth}{\paperwidth - 0.9847in-0.9847in}

  
\maketitle

\section{Introduction}
\subsection{What are word embeddings?}
Word Embedding are vector relationships between words in text which assign a value to semantic relationships between words.
There are many popular word embeddings toolkits available today. Perhpaps the most ubiquitous is Word2Vec \cite{NEEDED} and GloVe \cite{NEEDED}.
 Here you will identify the main problems - the problematic - of the research area.

\section{In Depth Review}
\subsection{Abstract}
\subsection{Problem Statement}
\subsection{Contributions}
\subsection{Approach}
\subsection{Methodology}
\subsection{Outcomes}
\subsection{Related Work}
\subsection{Conclusions}
\subsection{References}

\section{In Depth Review}
\subsection{Abstract}
\subsection{Problem Statement}
\subsection{Contributions}
\subsection{Approach}
\subsection{Methodology}
\subsection{Outcomes}
\subsection{Related Work}
\subsection{Conclusions}
\subsection{References}

\section{One Page Summary}
\subsection{Abstract}
\subsection{Summary}

\section{One Page Summary}
\subsection{Abstract}
\subsection{Summary}

\section{A Criticism of Word Embeddings in IR}
Word Embeddings Causes Topic \cite{rekabsaz2017word}
\section{Embedding Based Query Language Models}
In \cite{Zamani:2016:EQL:2970398.2970405} Zamani and Croft propose an application of word embeddings to help expand the terms from which a query language model can be created. A query language model is an information retrieval technique which allows one to model the language used in a query or document in order to create a statistical measure of how likely a word or phrase is to appear. A language model works through assigning probabilities to each word in the model and further, a probability of ending the phrase after each word.

Specifically they aim to apply word embeddings to help with "the vocabulary mismatch problem" which is the differing terminology used by users to describe the same concept.

Previous work has been done which addresses using word embeddings to help create document language models. Document language models are used to perform tasks such as generating query liklihood measures. That is, given a document language model, how likely is it that a given query would be generated by the language model. Documents which produce language models which are more likely to generate the given query are then weighted more heavily in retrieval.



   \subsection{Main problem 1}
   \subsection{Another important problem}
 \section{Language Modelling Outside of IR}
 
  An in depth presentation of the key ideas and results of the Bloggs/Smith
  school of thought\citep{SYMBOL}.
  
 \section{Early Language Modeling}
 
 How the startling results of Jones\citep{Hayes89} completely discredited Smith
 
 \section{Extensions to Language Modeling in IR}
 ..... As \citet{Einstein} says
 
 and so on until you have covered around half a dozen studies
 
\section{Recent Advances}

\section{Conclusions}
Weigh up the balance of evidence and arguments you have reviewed to say which
positions seem to have the greater empirical and or logical support.




% you should have a .bib file called mybibliographyfile.bib in the
% current directory for this to work
 \bibliographystyle{plainnat}
\bibliography{bibliography}







\end{document}
